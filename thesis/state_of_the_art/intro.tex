%\section{Introduction}

The historic evolution of computer vision might be seen as the evolution of the optimism of the investigators with respect to this field. 
\\

In the 1950s the first computers were being developed. There were many science-fiction films and books that suggested a new robotic era approaching. The optimism and the confidence in the power of computers were huge. 
Since seeing is easy for humans and the computers had a big capacity for computing, it was thought that the artificial vision field (as it was called in that period) will rapidly develop. 
\\

In the 1960s the investigators started to frustrate. They realized that, even though the hardware used for computer vision was better than the one the humans possess, the processing made by the mind was much better and complex than the algorithms developed at the time. 
The fact that the information received is a two-dimensional projection of a three-dimensional world requires that the processing step interpret and make an abstraction of that data. Most of the human mind is still a mystery, and since it is not known how it processes the information received by the eyes it is not possible to reproduce it in a software. 
At this time, it was discovered the importance of the learning stage in the recognition of objects. 
\\

The frustration and little advance in the field lasted two decades. It was in the 1980s when the investigators changed the way of thinking about the field. They changed the name to the one currently being used: "Computer Vision" and also changed the main investigation branches.
The evolution of the hardware (both computers and specialized hardware) also influenced the development of more complex projects. 
\\

In the following chapters the state of the art of the different parts that compose the project are presented. The present project is compared with previous ones from the most general to the most detailed point of view. 
