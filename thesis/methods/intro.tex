%\section{Introduction}

%In this chapter the conditions in which the software was tested and the elements present in that testing are explained. 
%Those elements are the computer and the RGB-D sensor being used. 
%The goal of this chapter is to detail the experiments that are performed in the system in order to allow others to replicate them. 
In this chapter the experiments that are performed in the system are detailed in order to allow others to replicate them. 
%\\%[0.5cm]

The software has been built in a modular way that allows an easy location of the problems and bottlenecks.
Different aspects of the code are tested and benchmarked such as the CPU and RAM consumption of the whole code. 
Also, since the computing has been distributed over the different nodes, the test describes the response of each one separately. 
The nodes are also benchmarked accordingly showing the publishing rate and the bandwidth. 
\\%[0.5cm]

Apart from the system's performance the accuracy is also being tested. 
This characteristic is measured taking into account the number of false positive and negatives that appear at the output of the system. 
The confusion matrix for each experiment is constructed. 
Using this information, the F1 score of the software is computed. 
% This study is conducted to determine the number of true and false positives and negatives.
% Using that data, in the following chapter the F-score of the software is calculated and the confusion matrix constructed. 
\\%[0.5cm]

%In the following sections different details of the experiments are detailed. 

