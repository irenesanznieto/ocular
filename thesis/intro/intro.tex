\chapter{Introduction}
The robotic field has experimented an enormous increase in the past years. 

Different components and systems have improved.
As an example, cameras have evolved and low-cost 3D sensors have appeared. 
This fact provides a higher and more reliable amount of information for the computer vision system of the robot.  
Also, the processing units have been upgraded, allowing a higher amount of computing that permits the inclusion of more complex algorithms in these robots.  
% Different aspects of the robots has improved such as the sensors or actuators being integrated. 
% The processing and analysis of external data has been also upgraded. 
This leads to the increasing introduction of robots in human environments with assistive or social tasks. 
\\


The first automatisms performed their tasks without being aware of their environment. 
Their work area consisted mainly on closed areas in which humans were not allowed. 
But the introduction of robots in the houses and other places frequented by humans increases the importance of the environment perception. 
% The integration of robots in areas inhabited with humans increase the importance of the environment perception. 
The robot must be able to recognize and interact with the objects and persons around it. 
This interaction needs many different sensor or input systems and as many output systems to retrieve the information and respond to it.  
\\

% Humans are a key part in that interaction. 
% Social and assistive robots are designed to help humans in their everyday life. 
% To this purpose, the robot should be able to recognize gestures or poses of the users or actions they are performing in order to infer what they are feeling or what they need. 
% As an example, if the user has fallen, the robot should recognize the pose to deduce the user's status. 
% Afterwards, it might be able to conclude that the user needs help from him in order to get up. 
% Finally the robot might offer a support or even help the user in the standing up action. 
% \\

The human environment has a lot of variables that contain information about the user's intentions or actions. 
One of this variables are the objects the human uses. 
For example, if a person is holding a toothbrush, he is probably going to brush his teeth. 
Also, if the user is holding the keys in his hand he might be going out of the house. 
It can be easily seen that identifying the objects the humans hold in their hands, key information about the actions they are undergoing is extracted. 
If the robots could recognize those objects, they could adapt their behaviour to the situation arount them. 
As an example, if the user is holding a book, he is probably reading and  he does not want to be disturbed but for an emergency. 
The robot could then change its behaviour to the new constrains of the environment around it, for example, making as less noise as possible. 
\\

Having a system that could track and identify in-hand objects could improve the robot's interaction with its environment.
For this purpose I have developed a system that is able to learn and recognize hand-held objects in real time through a simple and intuitive gesture interface. 
This software is capable of recognizing the user's skeleton and from it extract the hands position. 
With this information the system studies the area around the hands and compares it with the previously acquired dataset.
It is possible to use the software without input and output devices such a screen or a mouse, since it integrates a pose interface. 
This fact easies the transition from the dataset acquisition to the recognition of objects and allows the system to be implemented in a robot. 



% Robots are being increasingly introduced in human envionments. 
% Social and assistive robot's behaviour depend on the actions of the individuals around it. 
% Some of those actions may be inferred fro the object the user is holding. 
% For exmple, if a human is holding his keys, he is likely to go out. 
% An assistive robot for visually impaired people may respond to this action by handing him over his cane. 
% Recognizing the object the user is holding may help the robot to understand better the situation, improving the human-robot interaction. 
% For this purpose I have developed a system that is able to learn and recognize hand-held objects in real time through a simple and intuitive gesture interface. 

% \newpage 
%\addcontentsline{toc}{chapter}{Socio-economical context}
\section{Socio-economical context}

%Socio-economic - relating to both social and economic factors (social groups and the class system for example)
%Context - The circumstances/environment/events surrounding a specific thing. 




% \newpage


\section{Motivation}

Technology has evolved enormously in the past years. 
% In particular, robotics has changed and moved from controlled spaces such as factories to human inhabited spaces. 
% In section \ref{context} the reasons behind this shift in the robot's location and function were presented. 
% It was also estated that the importance of the perception systems augmented with the inclusion of robots in human-inhabited areas. 
% % This fact increases the importance of the perception systems being integrated. 
% The correct recognition of objects, persons, areas and situations is key for assistive and social tasks. 
Nowadays most of the robots being developed are only able to recognize a small part of their environment. 
% Most computer vision systems are currently used to navigate between points or to locate certain objects and grasp them. 
The decision algorithm of the robots is normally based on the instructions received from the user. 
Those commands are usually given by voice or inputting the desired task on a certain User Interface (UI).
\\

% For the robots to act as aiding personnel to take care of elder or sick people, the evolution of this decision algorithm is crucial. 
The algorithm decision of the robots must be upgraded if they are to perform tasks in a human inhabited area. %act as aiding personnel to take care of elder or sick people. 
They must be able to respond to commands, but also be aware of their environment and respond autonomously to it. 
% As an example, robots must be able to recognize the danger involved in certain objects or situations. 
% Also, they should perceive the environment and the interactions between humans to discern if, for example, they are arguing or if the conversation is not a beneficial one for the patient. 
% This latter example may be seen in the case of an anorexic person who is talking about food and weight with another person. 
% Then, the robot may intervene changing the subject subtly or trying to end the conversation. 
When this change has occurred, the robots may successfully perform the assistive tasks now reserved only to humans. 
But there is still a long way to research, mainly around the perception of the environment. 
The investment needed for its development is now being held due to the recession that appeared in the late 2000s decade, whose effects are nowadays still present. 
Nevertheless, this lack of funding might be mitigated using research, open data and open-source code. 
\\

I strongly believe in the ideals presented by the Open Source Initiative. 
This project is intended to be an open source code that can be a building block for other researchers that work on robot perception.
There are various scientists that have already studied the importance of the objects around the human in the action recognition. 
For example, in \cite{Delaitre}, a study of the human-object interaction in still images was performed in order to relate those objects to actions. 
	This relation between object and action may also be seen in \cite{Fathi}, in which wearable cameras are used to retrieve the input. 
	The objects being hand-held are recognized and the action associated to them is learned. 
	In this thesis I present a system that allows to easily learn and recognize hand-held objects in real time. 
	It is intended to be the previous step to the association between the object and the action. 



% The economic situation described above forced many experienced professionals to trying their luck creating start-ups. Many of the ideas of those enterprises are having nowadays a huge impact on the society. 
% \\

% One of these open-source projects that appeared was the low-cost 3D printers. These machines have changed the manner of investigating many fields, since they allow to design different pieces easily and have a 3D reproduction in a few hours. 
% \\

% Initially, they were used in investigation and more specifically in robotics, but now they are used in many different fields. Among those fields, there are medicine, construction or even food making. 
% \\

% In medicine, they have been a revolution since they allow to create customized and precise pieces in very few time. They have been used for prosthesis and implants for persons of various ages, even for babies. In the prosthesis fields in particular, the 3D printing technology is being a complete revolution. Before, the prosthesis were very expensive and permitted only fixed movements and combinations. The adaptations to each individual were made in the final product itself, trying to make it as comfortable as possible for the wearer. Nowadays, the prosthesis are customized for each patient, reducing the inconveniences and increasing their usability. Also, they can be easily and cheaply adapted for children as an example, who are still experimenting many changes in their bodies. They are much cheaper than they were before, and everyone with a 3D printer may construct one. 
% In order to 3D print a piece a file with its description is needed. There are many web-pages that store open-source designs that ranges from decoration models to complex prosthesis. This fact is decisive because there is not needed a huge amount of knowledge or money to improve the life quality of a person using these technologies. 
% \\

% There are numerous open-source projects and developers that put in common their knowledge to improve the technology being used. I have used many of them in the previous years, to learn about 3D printing, robotics or programming among other fields. 
% \\

% It is a fact that acquiring knowledge would be much difficult if the Open Source initiative has not been invented. This impulsed me towards developing something useful and that could be used by other people. The idea of creating a software that could be used in robotics investigation but also help people at the same time. 
% \\[0.5cm]

% Many of the projects are aimed at aiding physically impaired people, creating sternal skeletons and robotic arms that could aid them. But a personal fact led me to realize that visually impaired people were not having as much attention. The applications developed for them are still rough to use and also it is difficult for a grown person to develop his remaining senses to supply the information lost. 
% \\

% Besides, in the robotic field new lines of investigation have appeared. The social robots are now a reality and in the near future we will interact with them everyday. In order to understand the human behavior, the recognition of the objects being handled by them is crucial. 
% \\[0.5cm]


% Computer vision has experience an important improvement in the last years through the upgrade of the hardware and software that compose it. The hardware such as acquisition elements (cameras, depth sensors, etc) and computing elements (PCs or other programmable devices) have experienced a rapid advance in the past years. It allowed to process more data that is now obtained more accurately and with less noise. This increase in the computing power of the equipment created a possibility of introducing more complex libraries and frameworks and even operating systems. 
% \\

% Now, the technology is available to solve the problems presented, the aid of visually impaired people and the introduction of new information in the social robotics field. This is how the idea behind this thesis appeared: the creation of a modular software that implements an in-hand object recognition algorithm. 


%\newpage

\section{The solution}

In this thesis a new software has been developed that is intended to be used in robots as well as with visually impaired humans. The input of the system is an RGB-D sensor, which provides 3D information of the user located in front of it. 
\\

The requirements for both context are almost identical. The most important one is to have an intuitive human-machine interface. Also, the fast learning procedure is a requirement since the environment in the world is constantly changing and new objects may be needed to learn almost everyday. The quick recognition of objects is important as well in order to have an easy to use and comfortable software. 
\\

Having this requirements in mind, a gestural interface was designed that allows the passing of information from human to computer in a simple way. It can be seen that the software has two differentiated modes, the learning and the recognizing modes. The usability of the software is fully dependent on the easy transition for the user from one mode to the other. 
Furthermore, the humans can hold an object with either hand and the information of which hand is holding the object must be passed to the program. 


\begin{figure}[H]
	\centering
    \includegraphics[width=0.45\textwidth]{img/intro/learning.eps}
	\caption[Learning Mode Triggering]{Learning mode triggering using the gestural interface}
\end{figure}

The software is able to work with one hand at a time, and the hand that is located higher is the one being used. The gesture was designed to have the most comfortable and easy to maintain posture possible. In order to trigger the learning mode, the user only needs to extend his arm towards the sensor, like showing the item to it. This is a natural gesture usually performed between humans when introducing new objects to one another. Having the hand closer to the body the recognition mode is triggered and the program outputs the identification number of the object. 



\begin{figure}[H]
	\centering
    \includegraphics[width=0.45\textwidth]{img/intro/recognizing.eps}
	\caption[Recognizing Mode Triggering]{Recognizing mode triggering using the gestural interface}
\end{figure}


The present project has been designed to be as modular and reusable as possible making an easier task to develop complementary software such as handbags recognition or hats recognition with slightly modifications. Its structure consists on nodes that run on parallel and minimizes the lag due to the computing processes. It is an Open Source project, meaning that anyone can use and modify it.
\\

In order to facilitate the usage of the code it has been developed as a ROS \cite{ros} package. The source code and further installation instructions and license details might be found in the following link to the software's \href{http://github.com/irenesanznieto/ocular}{\color{blue}\underline {repository}}. 


%\newpage

\chapter {Regulatory compliance}

The present section covers the regulatory compliance that affects directly to the system presented. 
% It must be noted that since the project is a research project, most of the regulations here exposed may not be needed. 
% Nevertheless, they must be taken into account in the case that the system is commercialized.

	% \paragraph{ISO}\mbox{}\\

	% The International Organization for Standardization is an international organism that settles international standards. 
	% It has several joint committees with the International Electrotechnical Commission (IEC). 
	% They develop standards in different technical fields such as the electrical, electronic or IT fields. 
	% \\

	% The Information Technology (IT) term is related to the use of computers and telecommunications to interchange, store and manipulate data.  
	% It is a broad field in which the present project may be included.
	% The Joint Technical Committee devoted to this subject is the ISO/IEC JTC 1. 
	% Its main mission is develop, maintain, promote and facilitate IT standards regarding different areas such as the following : 
	% \begin{itemize}
	% 	\item{Design and development of systems and tools}
	% 	\item{Define performance and quality standards}
	% 	\item{Security, interoperability and portability of IT systems and products}
	% 	\item{Unified tools and environments as well as harmonized IT vocabulary}
	% \end{itemize}

	% This Committee has recently included the regulations regarding assistive tecnology (AT) \cite{japan1}\cite{japan2}.
	% %--> http://en.wikipedia.org/wiki/ISO/IEC_JTC_1 [IT projects] 


	% \paragraph{Software regulations}\mbox{}\\
	
	There are different regulations regarding software. 
	% This section is centered on the ones related to Open Source systems, since the software developed in this thesis is Open  
	Since the project is Open Source, the section is centered on presenting the regulations of this type of systems. 
	Nowadays, the author of the software has the right of sharing it using a contract. 
	In it he determines which of the author rights he is going to yield and under what conditions. 
	This type of contract is called a software license. 


	\section{OCULAR}

	This project involves taking personal information to construct a dataset with which recognize different objects. 
	In order to observe the data protection Spanish law (Ley Orgánica 15/1999, de 13 de diciembre, de Protección de Datos de Carácter Personal), the stored data does not contain any personal data. 
	Instead of storing an image, the information retrieved is a matrix of numbers containing the descriptors extracted from the objects. 
	This ensures that the system could be used in commercial software or that further investigations could be performed accordingly with the law. 

	This project is being distributed with a MIT License (MIT). 
	This license may be found in this  \href{https://raw.githubusercontent.com/irenesanznieto/ocular/master/LICENSE.md}{\color{blue}\underline {link}}, and states the following: \\

	"Copyright (c) 2014 Irene Sanz Nieto

Permission is hereby granted, free of charge, to any person obtaining a copy of this software and associated documentation files (the "Software"), to deal in the Software without restriction, including without limitation the rights to use, copy, modify, merge, publish, distribute, sublicense, and/or sell copies of the Software, and to permit persons to whom the Software is furnished to do so, subject to the following conditions:

The above copyright notice and this permission notice shall be included in all copies or substantial portions of the Software.

The software is provided "as is", without warranty of any kind, express or implied, including but not limited to the warranties of merchantability, fitness for a particular purpose and noninfringement. In no event shall the authors or copyright holders be liable for any claim, damages or other liability, whether in an action of contract, tort or otherwise, arising from, out of or in connection with the software or the use or other dealings in the software."
\\

	The last paragraph is published in uppercase letters, but it was converted to lowercase to avoid a disturbance in the structure of the thesis. 	
	The license claims that software is available for redistribution and use, but no warranty is provided with it. 
	The different algorithms and third-party packages were selected taking into account that their licenses must be compatible with the one being provided by this system. 
	In the next sections the licenses under which the different packages are distributed are presented. 


	\section{ROS}
	All ROS core code is distributed under a BSD license, more specifically a BSD 3-Clause license. 
	It is very similar to the OCULAR license, the redistribution is permitted under certain conditions. 
	More information may be found in this \href{http://opensource.org/licenses/BSD-3-Clause}{\color{blue} {webpage}}. 
	The different ROS packages that are used in this thesis (openni\_camera, openni\_launch and pi\_tracker) are distributed under a BSD license, according to their web pages ( \href{http://wiki.ros.org/openni_camera}{openni\_camera}, \href{http://wiki.ros.org/openni_launch}{openni\_launch}, \href{http://wiki.ros.org/pi_tracker}{pi\_tracker}).

	\section{OpenCV}
	The Open Source Computer Vision library is released as well under a BSD license. 
	Hence it is free for both academic and commercial use. 
	Nevertheless, there are certain algorithms implemented whose license is different from the whole library. 
	The SIFT or SURF descriptors are two examples of this fact. 
	These algorithms have a software patent. 
	This legal figure allows the use of the algorithms for investigation purposes. 
	But for commercial uses the payment of a fee is imposed. 
	This was one of the reasons of choosing the ORB algorithm as the descriptor extractor of the system. 
	ORB has no patent and hence could be used for both commercial and research. 

	\section{PCL}
	The Point Cloud Library has as well a BSD license. 
	It is then free for commercial and research use. 
	Further information about the library and its license may be found in this \href{http://pointclouds.org}{\color{blue} {webpage}}. 


	% \paragraph{Data protection} \mbox{}
	\\

	% Since this project implements a proof of concept, the experiments performed with it were done by the author. 
	

	% This law "warrants and protects the personal data treatment, the public liberties and the fundamental rights of the persons, specially their honor and personal and familiar intimacy".
	% It is applicable to all data stored in a physical support. 
% 	Privacidad y confidencialidad: cualquier investigación que contenga datos de caracter personal tiene que cumplir los preceptos de la legislación de protección de datos. En España la norma que regula estos aspectos es la Ley Orgánica 15/1999, de 13 de diciembre, de Protección de Datos de Carácter Personal, cuayo objeto es 'garantizar y proteger en lo que concierne al tratamiento de los datos personales, las libertades públicas y los derechos fundamentales de las personas físicas, y especialmente de su honor e intimidad personal y familiar”.
% La ley es de aplicación a los datos de carácter personal registrados en cualquier soporte físico. El tratamiento de los datos cubre las actividades de recolección, registro, almacenamiento, recuperación, consulta, uso y diseminación. Para garantizar el derecho a la protección de datos, es necesario informar a las personas implicadas y solicitar su consentimiento para el tratamiento de sus datos. 

%\newpage

\section{Objectives}

The objectives of this thesis are listed below. 

\begin{itemize}
	\item To develop a system capable of recognizing objects in real time. 
			By recognition it is meant the detection of new objects and the comparison with a previously obtained dataset outputting the ID number of the most similar template.
	\item To develop a system capable of learning objects in real time. 
			This includes the storage of the dataset for further usages.
			The learning process is performed acquiring a definable number of views per object. 
	\item The system must be able to detect the persons that are in front of the robot. 
	\item The software must detect the location of the hands of the user in order to extract from there the object to be recognized and learned. 
	\item The system must be able to learn more than one view of each object. 
	\item The software must have a gestural interface that allows to trigger the recognizing and the learning modes. 

\end{itemize}
%In order to achieve those objectives, an spiral software development model was applied. In this model, the software analysis, objectives definitions, prototypes creation and testing are done iteratively. 
\\

%The intermediate objectives of the thesis that were defined in each iteration are enumerated below. 

% \begin{itemize}

% 	\item{Develop or adapt an existing hand tracking software}
% 	\item{Develop a Region Of Interest extractor software in order to filter the input raw 2D information. }
% 	\item{Develop a Region Of Interest extractor software in order to filter the input raw 3D information. }
% 	\item{Develop an object learning software using 2D data that allows to extract multiple views from each object}
% 	\item{Develop an object learning software using 3D data that allows to extract multiple views from each object}
% 	\item{Develop an object recognition software using 2D data and multiple object's views}
% 	\item{Develop an object recognition software using 3D data and multiple object's views}
% 	\item{Develop a feedback system to reduce the recognition uncertainties}
% 	\item{Develop a gesture interface to control the software}
% 	\item{Compare the object recognition using 2D and 3D information, the effectiveness and efficiency of each algorithm}
% 	\item{Perform a comparative study when changing the number of views used per object}

% \end{itemize}


%\newpage

\addcontentsline{toc}{chapter}{Thesis structure}
\chapter*{ Thesis structure}

