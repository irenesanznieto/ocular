\section {Regulatory compliance}
The present section covers the regulatory compliance that affects directly to the system presented. 

	\paragraph{ISO}\mbox{}\\

	The International Organization for Standardization is an international organism that settles international standards. 
	It has several joint committees with the International Electrotechnical Commission (IEC). 
	They develop standards in different technical fields such as the electrical, electronic or IT fields. 
	\\

	The Information Technology (IT) term is related to the use of computers and telecommunications to interchange, store and manipulate data.  
	It is a broad field in which the present project may be included.
	The Joint Technical Comittee devoted to this subject is the ISO/IEC JTC 1. 
	Its main mission is develop, maintain, promote and facilitate IT standards regarding different areas such as the following : 
	\begin{itemize}
		\item{Design and development of systems and tools}
		\item{Define performance and quality standards}
		\item{Security, interoperability and portability of IT systems and products}
		\item{Unified tools and environments as well as harmonized IT vocabulary}
	\end{itemize}


	%--> http://en.wikipedia.org/wiki/ISO/IEC_JTC_1 [IT projects] 
	\paragraph{Software regulations}\mbox{}\\

	% --> Open Source software
	% [blablabla --> intro etc del pdf]



	The software created in this thesis is Open Source. 
	It is being distributed with a MIT License (MIT). 
	This license may be found in the repository and states the following: \\

	"Copyright (c) 2014 Irene Sanz Nieto

Permission is hereby granted, free of charge, to any person obtaining a copy of this software and associated documentation files (the "Software"), to deal in the Software without restriction, including without limitation the rights to use, copy, modify, merge, publish, distribute, sublicense, and/or sell copies of the Software, and to permit persons to whom the Software is furnished to do so, subject to the following conditions:

The above copyright notice and this permission notice shall be included in all copies or substantial portions of the Software.

THE SOFTWARE IS PROVIDED "AS IS", WITHOUT WARRANTY OF ANY KIND, EXPRESS OR IMPLIED, INCLUDING BUT NOT LIMITED TO THE WARRANTIES OF MERCHANTABILITY, FITNESS FOR A PARTICULAR PURPOSE AND NONINFRINGEMENT. IN NO EVENT SHALL THE AUTHORS OR COPYRIGHT HOLDERS BE LIABLE FOR ANY CLAIM, DAMAGES OR OTHER LIABILITY, WHETHER IN AN ACTION OF CONTRACT, TORT OR OTHERWISE, ARISING FROM, OUT OF OR IN CONNECTION WITH THE SOFTWARE OR THE USE OR OTHER DEALINGS IN THE SOFTWARE."

	
	As it can be seen, the software is available for redistribution and use, but no warranty is provided with it. 


	\paragraph{Data protection} 

	Privacidad y confidencialidad: cualquier investigación que contenga datos de caracter personal tiene que cumplir los preceptos de la legislación de protección de datos. En España la norma que regula estos aspectos es la Ley Orgánica 15/1999, de 13 de diciembre, de Protección de Datos de Carácter Personal, cuayo objeto es 'garantizar y proteger en lo que concierne al tratamiento de los datos personales, las libertades públicas y los derechos fundamentales de las personas físicas, y especialmente de su honor e intimidad personal y familiar”.
La ley es de aplicación a los datos de carácter personal registrados en cualquier soporte físico. El tratamiento de los datos cubre las actividades de recolección, registro, almacenamiento, recuperación, consulta, uso y diseminación. Para garantizar el derecho a la protección de datos, es necesario informar a las personas implicadas y solicitar su consentimiento para el tratamiento de sus datos. 