\section{Performance testing}

	\begin{itemize}
		\item{\textbf{Package and nodes benchmarking}}
		\\

			As it was said in chapter \ref{methods}, the benchmarking is made in the different nodes that compose the software. 
			There is a high difference in the CPU and RAM usage between nodes.
			The nodes that only perform a transformation of the data such as the converter node has a lower CPU and RAM consumption.
			The nodes that process the input images and point clouds have much higher values. 
			Figure \ref{node} shows the results of the node benchmarking. 

			\begin{figure}[h]
				\begin{center}
			    \includegraphics[width=\linewidth]{img/tests/node.png}
				\caption[Nodes benchmarking]{Nodes benchmarking}
				\label{node}
				\end{center}
			\end{figure}

			The total CPU usage is lower than the 23\%, and the RAM usage of the whole software is of less than the 5\%. 
			\\


			The difference between nodes is patent in the table.
			The CPU and RAM usage varies from  0.13 to 13.34 and from 0.1 to 1.5 respectively. 
			That is, there ia a percent variation of 99\% and  93\%  in the CPU and RAM consumption in the nodes. 
			The nodes with a higher computing consumption are the ROI segmenters and the feature extractors both 2D and 3D.
			\\

			The learner recognizer node also has a higher consumption than the converter, event handler or system output nodes. 
			
			\newpage

		\item{\textbf{Topic benchmarking}}\\

			The following figures show the publishing rate and bandwidth of the different topics. 
			In figure \ref{hz} can be appreciated the difference in the average rate between topics. 
			Also it is noticeable the minimum rate column. 
			Most of the topics have a minimum publishing rate of 0 seconds.
			The only one that is different is the final object ID topic. 
			Another singular fact is the maximum column. 
			In it, all topics but two have less than 1 second. 
			Those two are again final object ID and the object ID topic. 
			The reasons behind this particular behaviour are discussed in the next chapter. 

			\begin{figure}[h]
				\begin{center}
			    \includegraphics[width=\linewidth]{img/tests/topic_hz.png}
				\caption[Topic benchmarking - Publishing rate]{Topic benchmarking - Publishing rate}
				\label{hz}
				\end{center}
			\end{figure}

			The table below shows the bandwidth consumed. 
			It can be seen a huge difference between topics. 
			The data ranges from the bytes to megabytes. 
			The nodes using a bandwidth of bytes are segmented coordinates and final object ID. 
			They are closely followed by the hand location, descriptors 2D, event and object ID topics that are in the kilobytes range. 
			Finally, the ones using a higher bandwidth are the segmented image with and without keypoints, the segmented point cloud and the descriptors 3D topics. 
			\begin{figure}[h]
				\begin{center}
			    \includegraphics[width=\linewidth]{img/tests/topic_bw.png}
				\caption[Topic benchmarking - Bandwidth]{Topic benchmarking - Bandwidth}
				\label{bw}
				\end{center}
			\end{figure}

	\end{itemize}